\documentclass{article}

\title{The Art of Computer Programming}
\author{Eric Liu}
\date{\today}

\usepackage{amsmath}

\renewcommand{\O}{$\mathcal{O}$}

\begin{document}
	
	\section{Volume 1: Fundamental Algorithms, Third Edition}
	
		\subsection{Chapter 1.1: Algorithms}
		
			\paragraph{Exercise 1 [10]} TODO
			\paragraph{Exercise 2 [15]} TODO
			\paragraph{Exercise 3 [20]} TODO
			\paragraph{Exercise 4 [16]} TODO
			\paragraph{Exercise 5 [12]} TODO
			\paragraph{Exercise 6 [20]} TODO
			\paragraph{Exercise 7 [HM21]} TODO
			\paragraph{Exercise 8 [M25]} TODO
			\paragraph{Exercise 9 [M30]} TODO
			
		\subsection{Test Math}
			
			$$ x^n = \begin{Bmatrix} n \\ n \end{Bmatrix} x^n + \cdots + \begin{Bmatrix} n \\ 1 \end{Bmatrix} x^1 + \begin{Bmatrix} n \\ 0 \end{Bmatrix} x^0 = \sum_k \begin{Bmatrix} n \\ k \end{Bmatrix} x^k $$
			
		\subsection{Chapter 1.2.1: Mathematical Induction}
		
		\subsection{Chapter 1.2.2: Numbers, Powers, and Logarithms}
		
		\subsection{Chapter 1.2.3: Sums and Products}
		
		\subsection{Chapter 1.2.4: Integer Functions and Elementary Number Theory}
		
		\subsection{Chapter 1.2.5: Permutations and Factorials}
		
		\subsection{Chapter 1.2.6: Binomial Coefficients}
		
		\subsection{Chapter 1.2.7: Harmonic Numbers}
		
		\subsection{Chapter 1.2.8: Fibonacci Numbers}
		
		\subsection{Chapter 1.2.9: Generating Functions}
		
		\subsection{Chapter 1.2.10: Analysis of an Algorithm}
		
		\subsection{Chapter 1.2.11.1: The \O{}-Notation}
		
		\subsection{Chapter 1.2.11.2: Euler's Summation Formula}
		
		\subsection{Chapter 1.2.11.3: Some Asymptotic Calculations}
			
\end{document}