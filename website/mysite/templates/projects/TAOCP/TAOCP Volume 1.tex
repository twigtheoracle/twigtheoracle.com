\documentclass{report}

\usepackage{amsmath}

\renewcommand{\O}{$\mathcal{O}$}

\newcommand{\say}[1]{``#1''}

\setlength{\parindent}{0pt}

\begin{document}
	
\chapter*{Volume 1: Fundamental Algorithms, Third Edition}

	\section*{Chapter 1.1: Algorithms}
	
		\subsection*{Exercise 1 [10]} 
		
			The text showed how to interchange the values of variables $m$ and $n$, using the replacement notation, by setting $t \leftarrow m$, $m \leftarrow n$, $n \leftarrow t$. Show how the values of \textit{four} variables $(a,b,c,d)$ can be rearranged to $(b,c,d,a)$ by a sequence of replacements. Try to use the minimum number of replacements.
			
			\paragraph{Solution} I am 99\% sure that doing the above can be done in at minimum three switches, which I think are fairly obvious. 
			
			$a \leftrightarrow b$ to get $(b,a,c,d)$
			
			$c \leftrightarrow a$ to get $(b,c,a,d)$
			
			$a \leftrightarrow d$ to get $(b,c,d,a)$
			
			\paragraph{Corrections} I read the problem wrong, it was asking for $\leftarrow$ replacements instead of replacement $\leftrightarrow$.
			
			\paragraph{Solution} Basically, put use $t$ as a single placeholder variable and shift everything within around:
			
			$t \leftarrow a$ to get $(a,b,c,d)$ and $t=a$
			
			$a \leftarrow b$ to get $(b,b,c,d)$ and $t=a$
			
			$b \leftarrow c$ to get $(b,c,c,d)$ and $t=a$
			
			$c \leftarrow d$ to get $(b,c,d,d)$ and $t=a$
			
			$d \leftarrow t$ to get $(b,c,d,a)$
			
			\paragraph{Corrections} N/A
		
		\subsection*{Exercise 2 [15]} 
		
			Prove that $m$ is always greater than $n$ at the beginning of Step~E1, except possibly the first time this step occurs
			
			\paragraph{Solution} Either $m > n$, $m=n$, or $m < n$. For each of these situations:
			\begin{enumerate}
				\item Assume that $m > n$. In this case, the solution is trivial.
				\item Assume that $m=n$. We execute Step~E1 and get a remainder $r=0$. At Step~E2, the algorithm terminates so we never hit Step~E1 again.
				\item Assume that $m < n$. We execute Step~E1 and get the remainder of $r=m$. We execute Step~E3 and set $m \leftarrow n$ and $n \leftarrow r$. Now, $m$ has the value of the original $n$ and $n$ has the value of the original $m$. Since $m$ and $n$ have switched, we get the new relationship of $m > n$.
			\end{enumerate}
			
			\paragraph{Corrections} N/A
		
		\subsection*{Exercise 3 [20]} 
		
			Change Algorithm E (for the sake of efficiency) so that all trivial replacement operations such as \say{$m \leftarrow n$} are avoided. 
			
			\paragraph{Solution} The following algorithm eliminates all trivial replacement operations:
			
			\begin{enumerate}
				\item Divide $m$ by $n$ and let $m$ be the remainder
				\item If $m$ is zero, the algorithm terminates; $n$ is the answer
				\item Divide $n$ by $m$ and let $n$ be the remainder
				\item If $n$ is zero, the algorithm terminates; $m$ is the answer
				\item Return to step F1
			\end{enumerate}
			
			\paragraph{Corrections} N/A
			
		
		\subsection*{Exercise 4 [16]} 
		
			What is the GCD of 2166 and 6099?
			
			\paragraph{Solution} For fun, let's use the algorithm in \textbf{Exercise 3} with $m=2166$ and $n=6099$.
			
			\begin{table}[h]
				\begin{tabular}{llll}
					$m$ & $n$ & $m\%n$ & $n\%m$ \\
					\hline
					2166 & 6099 & 2166 & \\
					2166 & 6099 & & 1767 \\
					2166 & 1767 & 399 & \\
					399 & 1767 & & 171 \\
					399 & 171 & 57 & \\
					57 & 171 & & 0 \\
					57 & 0 & 0 & 
				\end{tabular}
			\end{table}
		
			So the GCD is 57
			
			\paragraph{Corrections} N/A
			
		
		\subsection*{Exercise 5 [12]} 
		
			Show that the \say{Procedure for Reading This Set of Books} that appear after the preface actually fails to be a genuine algorithm on at least three of our five counts. Also mention some differences in format between it an Algorithm E.
			
			\paragraph{Solution} The five counts are as follows:
			
			\begin{enumerate}
				\item Finiteness: An algorithm must always terminate after a finite number of steps.
				\item Definiteness: Each step of an algorithm must be precisely defined; the actions to be carried out must be rigorously and unambiguously specified for each case.
				\item Input: An algorithm has zero or more inputs, quantities that are given to it initially before the algorithm begins, or dynamically as the algorithm runs.
				\item Output: An algorithm has one or more outputs, quantities that have a specified relation to the inputs
				\item Effectiveness: An algorithm is also generally expected to be effective, in the sense that its operations must all be sufficiently basic that they can in principle be done exactly and in a finite length of time by someone using pencil and paper.
			\end{enumerate}
			
			The \say{Procedure for Reading This Set of Books} fails on counts 1, 2, and 4. The procedure is not finite because all 17 steps before the last either point to another step or move on to the next. The final step returns you to step 3, so the procedure will never end. The procedure is not definite because the steps leave some things to the discretion of the reader, which is the opposite of a precise definition. And finally, the procedure has no output, because it never ends.
			
			And here are a few differences with Algorithm E:
			\begin{itemize}
				\item The procedure lacks a formal end point, where Algorithm E uses a heavy vertical line
				\item The procedure is informal and vague, while Algorithm E uses strict notation and language
			\end{itemize}
			
			\paragraph{Corrections} N/A
			
		
		\subsection*{Exercise 6 [20]}
		
			What is $T_5$, the average number of times step E1 is performed when $n=5$.
			
			\paragraph{Solution} Let's compute some values for $m \in {1,2,3,4,5}$. According to the book, to find $T_n$ we just do the algorithm for $m=1$ to $m=n$, count the number of times step E1 has been done, and divide by $n$
			
			\begin{itemize}
				\item For $n=5$ and $m=1$. E1: $1\%5 = 1$. E3: $n=1$ and $m=5$. E1: $5\%1=0$. End with 2
				
				\item For $n=5$ and $m=2$. E1: $2\%5 = 2$. E3: $n=2$ and $m=5$. E1: $5\%2=1$. E3: $n=1$ and $m=2$. E1: $2\%1 = 0$ End with 3
				
				\item For $n=5$ and $m=3$. E1: $3\%5 = 3$. E3: $n=3$ and $m=5$. E1: $5\%3=2$. E3: $n=2$ and $m=3$. E1: $3\%2=1$. E3: $n=1$ and $m=1$. E1: $1\%1=0$. End with 4
				
				\item For $n=5$ and $m=4$. E1: $4\%5 = 1$. E3: $n=1$ and $m=5$. E1: $5\%1=0$. End with 2
				
				\item For $n=5$ and $m=5$. E1: $5\%5 = 0$. End with 1
			\end{itemize}
		
			The total number of times step E1 has been done is $2+3+4+2+1=12$ and divide by $n=5$ to get 2.4 
			
			\paragraph{Corrections} Silly mistake in $m=4$. 
			
			\paragraph{Solution} It should be:
			
			For $n=5$ and $m=4$. E1: $4\%5 = 4$. E3: $n=4$ and $m=5$. E1: $5\%4=1$. E3: $n=1$ and $m=4$. E1: $4\%1=0$. End with 3
			
			So the total should be $2+3+4+3+1=13$ and divide by $n=5$ to get 2.6
			
			As some additional food for thought, why do we only need to look at values of $m$ up to $n$? Let's say $m=6$. Then E1: $6\%5=1$, E3: $n=1$, $m=5$. This is identical to $m=1$, a pattern which will repeat for all values of $m > n$. Essentially, if $m\%n=x$, the steps followed are identical for if $m$ had equaled $x$ to start with.
			
			\paragraph{Corrections} N/A
			
		
		\subsection*{Exercise 7 [HM21]} 
		
			Let $U_m$ be the average number of times that step E1 is executed in Algorithm E, if $m$ is known and $n$ is allowed to range over all positive integers. Show that $U_m$ is well defined. Is $U_m$ in any way related to $T_m$?
			
			\paragraph{Solution} Let's first try some examples I wrote some Python code (TODO: link) to do the computation for values of $n$ from 1 to 1,000,000 with the following results:
			
			\begin{table}[h]
				\begin{tabular}{lllll}
					$m$ & $U_m$ & Approximation & Expanded Numerator \\
					\hline
					1 & 1.999999 & 2 & 2 \\
					2 & 2.499997 & $2\frac{1}{2} = 2.5$ & 5 \\
					3 & 2.999995 & 3 & 9 \\
					4 & 2.999993 & 3 & 12 \\
					5 & 3.599991 & $3\frac{3}{5} = 3.6$ & 18 \\
					6 & 3.166656 & $3 \frac{1}{6} \approx 3.1\overline{6}$ & 19\\
					7 & 3.857129 & $3 \frac{6}{7} \approx 3.85714\dots$ & 27 \\
					8 & 3.749985 & $3 \frac{6}{8} = 3.75$ & 30\\
					9 & 3.77776 & $3 \frac{7}{9} \approx 3.\overline{7}$ & 34\\
					10 & 3.699981 & $3 \frac{3}{10} = 3.7$ & 33\\
					11 & 4.454523 & $4 \frac{5}{11} \approx 4.\overline{45}$ & 49 \\
					12 & 3.666641 & $3 \frac{8}{12} \approx 3.\overline{6}$ & 44 \\
					13 & 4.461512 & $4 \frac{6}{13} \approx 4.46153\dots$ & 58 \\
					14 & 4.21425 & $4 \frac{3}{14} \approx 4.21428\dots$ & 59 \\
					15 & 4.066637 & $4 \frac{1}{15} \approx 4.0\overline{6}$ & 61 \\
					16 & 4.124969 & $4 \frac{2}{16} = 4.125$ & 66 \\
					17 & 4.705847 & $4 \frac{12}{17} \approx 4.70588\dots$ & 80 \\
					18 & 4.27774 & $4 \frac{5}{18} = 4.2\overline{7}$ & 77 \\
					19 & 4.894699 & $4 \frac{17}{19} \approx 4.8947\dots$ & 93 \\
					20 & 4.199961 & $4 \frac{4}{20} = 4.2$ & 84 \\
				\end{tabular}
			\end{table}
			
			Note, based on observation of $U_m$ as the maximum value of $n$ increases, I am very confident that the value of $U_m$ will always converge. Additionally, the problem statement implies that $U_m$ is well defined.
			
			
			\paragraph{Corrections} TODO
			
		
		\subsection*{Exercise 8 [M25]} 
		
			Give an \say{effective} formal algorithm for computing the GCD of positive integers $m$ and $n$ by specifying $\theta_j, \phi_j, a_j, b_j$ as in Eqs. (3). Let the input be represented by the string $a^m b^n$, that is, $m$ $a$'s followed by $n$ $b$'s. Try to make your solution as simple as possible. [\textit{Hint:} Use Algorithm E, but instead of division in step E1, set $r \leftarrow |m-n|$, $n \leftarrow \text{min}(m,n)$.]
			
			\paragraph{Solution} TODO
			
			\paragraph{Corrections} TODO
			
		
		\subsection*{Exercise 9 [M30]} 
		
			Suppose that $C_1 = (Q_1, I_1, \Omega_1, f_1)$ and $C_2 = (Q_2, I_2, \Omega_2, f_2)$ are computational methods. For example $C_1$ might stand for Algorithm E as in Eqs. (2), except that $m$ and $n$ are restricted in magnitude, and $C_2$ might stand for a computer program implementation of Algorithm E. (Thus $Q_2$ might be the set of all states of the machine, i.e., all possible configurations of its memory and registers; $f_2$ might be the definition of single machine actions; and $I_2$ might be the set of initial states, each including the program that determines the GCD as well as the particular values of $m$ and $n$.)
			
			Formulate a set-theoretic definition for the concept \say{$C_2$ is a representation of $C_1$} or \say{$C_2$ simulates $C_1$}. This is to mean intuitively that any computation sequence of $C_1$ is mimicked by $C_2$, except that $C_2$ might take more steps in which to do the computation and it might retain more information in its states
			
			\paragraph{Solution} TODO
			
			\paragraph{Corrections} TODO
			
		
		
%	\subsection{Test Math}
%		
%		$$ x^n = \begin{Bmatrix} n \\ n \end{Bmatrix} x^n + \cdots + \begin{Bmatrix} n \\ 1 \end{Bmatrix} x^1 + \begin{Bmatrix} n \\ 0 \end{Bmatrix} x^0 = \sum_k \begin{Bmatrix} n \\ k \end{Bmatrix} x^k $$
%		
%	\subsection{Chapter 1.2.1: Mathematical Induction}
%	
%	\subsection{Chapter 1.2.2: Numbers, Powers, and Logarithms}
%	
%	\subsection{Chapter 1.2.3: Sums and Products}
%	
%	\subsection{Chapter 1.2.4: Integer Functions and Elementary Number Theory}
%	
%	\subsection{Chapter 1.2.5: Permutations and Factorials}
%	
%	\subsection{Chapter 1.2.6: Binomial Coefficients}
%	
%	\subsection{Chapter 1.2.7: Harmonic Numbers}
%	
%	\subsection{Chapter 1.2.8: Fibonacci Numbers}
%	
%	\subsection{Chapter 1.2.9: Generating Functions}
%	
%	\subsection{Chapter 1.2.10: Analysis of an Algorithm}
%	
%	\subsection{Chapter 1.2.11.1: The \O{}-Notation}
%	
%	\subsection{Chapter 1.2.11.2: Euler's Summation Formula}
%	
%	\subsection{Chapter 1.2.11.3: Some Asymptotic Calculations}
			
\end{document}